\documentclass[]{report}
\usepackage{a4wide}
\usepackage{amsmath}


% Title Page
\title{PDEs assignment}
\author{Robin Evers}


\begin{document}
\maketitle
\begin{itemize}
	\item[1.] Draw the characteristics for the following equations and write the general solution.
	\begin{align}
		u_t + 7u_x &= 0\\
		u_t - 2u_x + 3u &= 0.
	\end{align}
	\item[Sol.] To find characteristics, we solve the ODE
	\begin{align*}
		x'(t)&=7\\
		x(0)&=x_0.
	\end{align*} 
	This gives us the characteristic curves $x(t)$ defined by 
	\[x(t)=x_0+7t.\]
	On these curves we have
	\begin{align*}
	\frac{d}{dt}u(x(t),t)&=(u(x(t),t))_t+(x(t))_t(u(x(t)),t)_x\\
	&=(u(x(t),t))_t+7(u(x(t)),t)_x\\
	&=0.
	\end{align*}
	This means that solutions $u(x(t),t)$ to (1) are constant on these characteristic curves $x(t)$, hence the general solution to (1) is given by
	\[u(x(t),t)=u(x_0,0)=u(x(t)-7t,0).\]
	\item[Sol.] To find characteristics, we solve the ODE
	\begin{align*}
	x'(t)&=-2\\
	x(0)&=x_0.
	\end{align*} 
	This gives us the characteristic curves $x(t)$ defined by 
	\[x(t)=x_0-2t.\]
	On these curves we have
	\begin{align*}
	\frac{d}{dt}u(x(t),t)&=(u(x(t),t))_t+(x(t))_t(u(x(t)),t)_x\\
	&=(u(x(t),t))_t-2(u(x(t)),t)_x\\
	&=-3u(x(t),t).
	\end{align*}
	Hence
	\[\frac{d}{dt}u(x(t),t)+3u(x(t),t)=0.\]
	This gives us the ODE
	\begin{align*}
		u_t+3u&=0\\
		u(x(0),0)&=u_0(x_0).
	\end{align*} 
	We multiply the ODE by $e^{3t}$ to get
	\[\frac{d}{dt}ue^{3t}=u_te^{3t}+3ue^{3t}=0.\]
	This means that $ue^{3t}$ is constant on $x(t)$, so
	\[e^{3t}u(x(t),t)=e^0u(x(0),0)=u_0(x_0)\]
	hence the general solution to (2) is given by
	\[u(x(t),t)=e^{-3t}u_0(x_0)=e^{-3t}u_0(x(t)+2t).\]
	\item[2.] Let $g$ be the gravitational constant and $h : \mathbf{R} \times \mathbf{R}^{+}$ be the height of a fluid, define $\phi = gh$. Further, let $v : \mathbf{R} \times \mathbf{R}^{+}$ be the horizontal velocity. The shallow water equations are
	given by
	\begin{equation}
	\begin{split}
		\phi_t +(v\phi)_x &= 0\\
		v_t + \left(\frac{v^2}{2}+\phi\right)_x&=0.
	\end{split}
	\end{equation}
	Let $\eta = \eta(\phi, v)$ be an entropy for the shallow water equations. Prove that it must satisfy
	\begin{equation}
		\frac{\partial^2\eta}{\partial v^2} = \phi \frac{\partial^2\eta}{\partial\phi^2}.
	\end{equation}
	\item[Sol.] We start by rewriting the shallow water equations as in (4) to one vectorial equation
	\[\partial_t\vec{u}+\partial_x \vec{f}(\vec{u})=0\]
	where $\vec{u}=(\phi,v)$ and $\vec{f}=(v\phi,\frac{v^2}{2}+\phi)$. The Jacobian of $f$ is given by
	\begin{align*}
		D\vec{f}=\begin{pmatrix}
		\frac{\partial f_1}{\partial \phi}	& \frac{\partial f_1}{\partial v}\\
		\frac{\partial f_2}{\partial \phi}	& \frac{\partial f_2}{\partial v}\end{pmatrix}
		=\begin{pmatrix}
		v	& \phi\\
		1	& v
		\end{pmatrix}.
	\end{align*}
	Let $\eta = \eta(\phi, v)$ be an entropy for the shallow water equations, the matrix of second derivatives of $\eta$ is given by
	\begin{align*}
	D^2\eta=\begin{pmatrix}
	\frac{\partial^2 \eta}{\partial \phi^2}	& \frac{\partial^2 \eta}{\partial \phi \partial v}\\
	\frac{\partial^2 \eta}{\partial v \partial \phi}	& \frac{\partial^2  \eta}{\partial v^2}\end{pmatrix}.
	\end{align*}
	Since $\eta$ is an entropy for the shallow water equations, we have
	\[(D\vec{f})^T D^2\eta = D^2 \eta D\vec{f}.\]
	This gives us four equations, the equation given by the second row, first column is
	\[\phi \frac{\partial^2 \eta}{\partial \phi^2}	+ v  \frac{\partial^2 \eta}{\partial v \partial \phi} =  v  \frac{\partial^2 \eta}{\partial v \partial \phi} + \frac{\partial^2 \eta}{\partial v^2}.\]
	By subtracting $v  \frac{\partial^2 \eta}{\partial v \partial \phi}$ from both sides we find the requested equality for our entropy $\eta$
	\newpage
	\item[3.]The Barotropic compressible Euler equations are given when the internal energy
	of the system is constant. The equations are given by conservation of mass and momentum
	\begin{equation}
	\begin{split}
		\rho_t +(\rho v)_x &= 0\\
		(\rho v)_t + \left(\rho v^2 + p\right)_x &= 0,
	\end{split}
	\end{equation}
	with $p = p(\rho)$ for a smooth function $p$. Assume $p'(\rho)>0$ then show that
	\begin{equation}
	\eta(\rho,\rho v)= \frac12 \rho v^2 + P(\rho)
	\end{equation}
	is an entropy when $P''(\rho) = \frac{p'(\rho)}{\rho}$ for $\rho >0$. What is the associated entropy flux?
	\item[Sol.]We start by rewriting the Euler equations as in (5) to one vectorial equation
	\[\partial_t\vec{u}+\partial_x \vec{f}(\vec{u})=0\]
	where $\vec{u}=(\rho, \rho v)$ and $\vec{f}(y,z)=(z,\frac{z^2}{y}+p(y))$. 
	The Jacobian of $f$ is given by
	\begin{align*}
	D\vec{f}=\begin{pmatrix}
	\frac{\partial f_1}{\partial y}	& \frac{\partial f_1}{\partial z}\\
	\frac{\partial f_2}{\partial y}	& \frac{\partial f_2}{\partial z}\end{pmatrix}
	=\begin{pmatrix}
	0	& 1\\
	p'(y) - (\frac{z}{y})^2	& \frac{2z}{y}
	\end{pmatrix}.
	\end{align*}
	We write $\eta$ as a function of variables $y, z$ as $\eta(y,z)=\frac{z^2}{2y}+P(y)$. The Jacobian of $\eta$ is given by
	\begin{align*}
	D\eta=\begin{pmatrix}
	\frac{\eta}{\partial y}	,& \frac{\partial \eta}{\partial z}\end{pmatrix}
	=\begin{pmatrix}
	\frac{-z^2}{2y^2} + P'(y)	,& \frac{z}{y}
	\end{pmatrix}.
	\end{align*}
	A pair $(\eta,q)$ is an entropy/entropy flux pair associated with (5) if $Dq=D\eta D\vec{f}$.
	If $\eta$ as defined in (6) is an entropy, we find that the entropy flux $q$ must satisfy
	\begin{align*}
			Dq=D\eta D\vec{f}&=\begin{pmatrix}
			\frac{-z^2}{2y^2} + P'(y)	,& \frac{z}{y}
			\end{pmatrix} \begin{pmatrix}
		0	& 1\\
		p'(y) - (\frac{z}{y})^2	& \frac{2z}{y}
		\end{pmatrix}\\
		&=\begin{pmatrix}
		\frac{zp'(y)}{y} - (\frac{z}{y})^3	,& \frac{3z^2}{2y^2}+P'(y).
		\end{pmatrix}\\
	\end{align*}
	Hence the partial derivative of $q$ to $z$ has to be
	\[\frac{\partial q}{\partial z} = \frac{3z^2}{2y^2}+P'(y) = \frac{3}{2y^2} z^2 + P'(y),\]
	which means $q$ has to be of the form
	\[q=\frac{1}{2y^2}z^3+P'(y)z + C(y).\]
	where $C$ is independent of $z$. We derive this to $y$ to find
	\[\frac{\partial q}{\partial y} = -\left(\frac{z}{y}\right)^3 + P''(y)z + C'(y). \]
	This equals the expression we found for $Dq$ if $P''(y) = \frac{p'(y)}{y}$ and $C'(y)=0$. Therefore $\eta(\rho, \rho v)$ is an entropy when $P''(\rho) = \frac{p'(\rho)}{\rho}$ for $\rho>0$, the associated entropy flux is given by 
	\[q=\frac{(\rho v)^3}{2\rho^2}+P'(\rho)\rho v + C = \frac{\rho v^3}{2} + P'(\rho)\rho v + C,\]
	where $C$ is independent of $\rho$ and $v$.
	\newpage
	\item[4.]
	Consider the following partial differential equation (PDE)
	\begin{equation}
	u_t + u_{xxx} + 6uu_x = 0, u = u(x, t)
	\end{equation}
	known as the Korteweg-de Vries equation (or KdV) and which describes water waves in shallow waters.
	\begin{itemize}
		\item[(a)] Consider a scaling transformation, on both the independent and dependent
		variables, of the form
		\begin{equation}
		S : (t, x, u) \mapsto (T, X, U) = (at, bx, cu)
		\end{equation}
		where $a, b$ and $c$ are nonzero constants. Find conditions for the parameters $a, b$ and $c$
		such that the transformation $S$ is a symmetry transformation of the KdV equation. Use the obtained symmetry to deduce that if $u = f(x, t)$ is a solution of the KdV equation then $u = \epsilon^2f(\epsilon x, \epsilon^3 t)$ is also a solution for all nonzero $\epsilon$.
		\item[Sol.] The PDE given in (6) admits the symmetry if, given a solution $u = f(x, t)$, the transformed function \[\tilde{u} = \tilde{f}(\tilde{x},\tilde{t})=cf(bx,at)\] 
		is also a solution of the same PDE, i.e.
		\begin{equation}
			\tilde{u}_{\tilde{t}} + \tilde{u}_{\tilde{x}\tilde{x}\tilde{x}} + 6\tilde{u}\tilde{u}_{\tilde{x}} = 0.
		\end{equation}
		To find $a,b,c$ for which this is true, we first calculate the transformed derivatives
		\[\frac{\partial}{\partial\tilde{t}}\tilde{u}(\tilde{x},\tilde{u})=\frac{dt}{d\tilde{t}}\frac{\partial}{\partial t}cu(x,t)=\frac{c}{a}\frac{\partial}{\partial t} u(x,t),\]
		
		\[\frac{\partial}{\partial\tilde{x}}\tilde{u}(\tilde{x},\tilde{u})=\frac{dx}{d\tilde{x}}\frac{\partial}{\partial x}cu(x,t)=\frac{c}{b}\frac{\partial}{\partial x} u(x,t),\]
		and similarly
		\[\frac{\partial^2}{\partial\tilde{x}^2}\tilde{u}(\tilde{x},\tilde{t})=\frac{c}{b^2}\frac{\partial^2}{\partial x}u(x,t),\]
		\[\frac{\partial^3}{\partial\tilde{x}^2}\tilde{u}(\tilde{x},\tilde{t})=\frac{c}{b^3}\frac{\partial^3}{\partial x}u(x,t).\]
		Hence, if we substitute to the left hand side of (8) we obtain
		\[\frac{c}{a}\frac{\partial}{\partial t} u(x,t)+\frac{c}{b^3}\frac{\partial^3}{\partial x}u(x,t) + \frac{6c^2}{b}u(x,t)\frac{\partial}{\partial x} u(x,t)=0\]
		when
		\[\frac{c}{a}=\frac{c}{b^3}=\frac{c^2}{b}.\]
		Since we assume $a,b,c$ nonnegative, this holds when $a=b^3$ and $c=b^2$. Let us fix $b=\epsilon$ for any nonnegative $\epsilon$, then 
		\[\tilde{u} = cf(bx,at) = \epsilon^2f(\epsilon x, \epsilon^3)\]
		is also a solution to (6). 
		\item[(b)]
	\end{itemize}

\end{itemize}

\end{document}          
